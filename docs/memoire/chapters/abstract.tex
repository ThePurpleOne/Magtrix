%% Tex spellcheck = fr_FR
\thispagestyle{noheader}
\chapter*{Abstract} % No (numbered) toc entry with *

\tikz[remember picture,overlay] \node[shift={(4.165cm,-1.955cm)}]
at (current page.north west)
{\includegraphics[height=1.29cm]{template/images/title/hepia_logo}};
\tikz[remember picture,overlay] \node[shift={(-4.238cm,-1.97cm)}]
at (current page.north east)
{\includegraphics[height=1.29cm]{template/images/title/hes-so_geneve_logo}};

\addcontentsline{toc}{chapter}{Abstract} % Adding toc entry
\thispagestyle{noheader}

\begin{spacing}{0.956}
	\vspace{0.5cm}

	In the era of autonomous drones and \gls{uav}, there is a pressing need among constructors and major technology firms to establish a robust testing framework that ensures safety, reliability, cost-effectiveness, repeatability, and control, all while providing easily quantifiable metrics. This project endeavors to meet these demands by developing a sophisticated \gls{gnss} spoofer utilizing \gls{sdr} technology. The primary objective is to fabricate fake \gls{gnss} signals capable of deceiving drones into perceiving forward movement, even when they are stationary in front of an artificial obstacle, such as a fan wall. The overarching goal of this system is to create a controlled testing environment for \gls{uav}, facilitating the assessment and quantification of their flying performance metrics. By employing high-frequency position tracking alongside precise control over wind speed and \gls{gnss} signals, the system enables \gls{uav} to hover in place while experiencing simulated forward motion. This setup provides a comprehensive means of evaluating the \gls{uav}'s response and performance under dynamic conditions, without the need for costly and potentially hazardous outdoor testing scenarios. The proposed system promises to revolutionize \gls{uav} testing procedures by offering a safe, repeatable, and cost-effective solution that can be tailored to specific testing requirements. Moreover, its ability to generate easily quantifiable metrics ensures that developers and engineers can accurately assess the performance of \gls{uav} across various flight scenarios. Ultimately, this innovation holds the potential to accelerate advancements in autonomous drone technology by providing a reliable testing platform that accurately reflects real-world challenges.

	\vfill
	\begin{center}
		{\includegraphics[width=0.5\linewidth]{figures/windshaper.png}}\\*
		\vfill
		%% CONTENT ENDS HERE

		{
			%%%%%%%%%%%%%%%%%%%%%%%%%%%%%%%%%%%%%%%%%%%%%%%%%%%%%%%%%%%%%%%%%%%%%%%%%%%%%%%%
			%%%%%%%%%%%%%%%%%%%%%%%%%% DO NOT MODIFY THE TABLE BELOW %%%%%%%%%%%%%%%%%%%%%%%
			%%%%%%%%%%%%%%%%%%%%%%%%%%%%%%%%%%%%%%%%%%%%%%%%%%%%%%%%%%%%%%%%%%%%%%%%%%%%%%%%
			\begin{tabular*}{16cm}{p{7.59cm} p{7.58cm}}
				\small Candidate:					&	\small Professor:\\*[10pt]
				\small\textbf{\textsc{\Author}}		&	\small\textbf{\textsc{\Professor}}\\*[10pt]
				\footnotesize  Branch : ISC	&	\footnotesize  \textbf{In collaboration with:} Windshape SA \\*[10pt]
				\footnotesize  {} & \footnotesize  Thesis subject to an internship agreement: \Convention\\*[20pt]
				\footnotesize  {} & \footnotesize  Work subject to confidentiality agreement: \Confidentiel\\*[10pt]
			\end{tabular*}\\*[1.9cm]
		}

	\end{center}
\end{spacing}
