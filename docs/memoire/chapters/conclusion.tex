%% Tex spellcheck = fr_FR
\chapter*{Conclusion}
\addcontentsline{toc}{chapter}{Conclusion} % Adding toc entry

The project set out to explore the potential applications of the USRP B200 in spoofing a commercial \gls{gps} receiver. Following an extensive research phase, we conducted experiments to assess the project's viability. Initially, our experiments involved comparing the performance of a commercial \gls{gps} receiver with that of the \gls{sdr} in obtaining a position fix. As anticipated, the commercial receiver performed admirably, while the \gls{sdr} exhibited lower accuracy, likely due to factors such as its algorithm or atmospheric correction model.

This project proved to be a fascinating learning journey, particularly in delving into RF, GNSS, and \gls{sdr} technologies, which were previously unfamiliar to me despite my background in electronics. The experience of applying new concepts in a practical setting and conducting thorough research was both rewarding and enlightening.

In the end, our efforts bore fruit as we successfully obtained a position fix using the \gls{sdr} and effectively spoofed a commercial receiver, thus demonstrating the feasibility of our approach. Looking ahead, there are opportunities for further refinement and enhancement of the system. These may include incorporating features such as support for multiple constellations and frequencies to enhance compatibility with a wider range of UAVs. Moreover, there is potential in developing a user-friendly interface to streamline control and operation, as well as integrating the spoofing system into Windshape's testing environment for comprehensive evaluation and validation.

By continuing to build upon the foundation established in this project, we can unlock even greater possibilities in the field of \gls{gps} spoofing and its applications.

