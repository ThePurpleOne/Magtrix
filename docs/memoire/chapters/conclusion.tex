%% Tex spellcheck = fr_FR
\chapter*{Conclusion}
\addcontentsline{toc}{chapter}{Conclusion} % Adding toc entry


This thesis presents a proof of concept for a scalable, automated board game system that integrates PCB-based electromagnetics, offering a novel approach to bringing digital game tracking into the physical world. Initially focused on a Tic-Tac-Toe game, the project successfully demonstrated the ability to move game pieces automatically via coils etched directly onto the PCB, controlled by a web application.

The iterative process of coil design and control system development highlighted the potential of this technology to be adapted for more complex games like chess, with minimal adjustments. The final prototype, a 3x3 Tic-Tac-Toe board, effectively showcases the concept’s viability, despite some hardware imperfections.

While the current implementation is limited to Tic-Tac-Toe, the core technology is designed with scalability in mind, offering possibilities for expansion into larger and more intricate board games.

The simulation of coil designs proved challenging, as I was unable to achieve results that matched real-world outcomes. This discrepancy primarily came from limited expertise in electromagnetism and the inherent complexity of the simulations. As a result, the design process relied heavily on a trial-and-error approach. A more in-depth theoretical understanding of electromagnetism would likely have led to more accurate simulations and a more efficient identification of the optimal coil design. Moving forward, strengthening the theoretical foundation in this area could greatly enhance the design process, leading to more precise and reliable outcomes.