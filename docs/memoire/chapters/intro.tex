%% Tex spellcheck = fr_FR
\chapter*{Introduction}
\addcontentsline{toc}{chapter}{Introduction} % Adding toc entry

During our final year of the Bachelor's program in Computing and Communication Systems at \gls{hepia}, we undertake a semester-long project aimed at providing hands-on experience with the tools and technologies relevant to our field. This project serves as a precursor to our Bachelor thesis, allowing us to explore areas of interest and gain practical skills. This project is done in collaboration with Windshape\footnote{\url{https://windshape.com/}}, a company that designs, manufactures and operates wind facilities. The company's objective is to offer comprehensive solutions for both aerodynamic research and drone certification. The project detailed in this report focuses on developing a cost-effective, modular and reliable \gls{gnss} spoofer using a \gls{sdr}. This spoofer will play a critical role in a fully controlled testing environment, where we could meticulously assess and certify the performance of \gls{uav}s.

The technical challenges of this project are numerous and varied, ranging from understanding the \gls{gnss} signal structure to implementing a spoofer capable of generating signals that are indistinguishable from the real ones. This report will only go as far as the proof of concept, but the project will continue in the form of a Bachelor thesis.

The methodology employed in this project began with an initial phase of familiarization with existing \gls{gnss} spoofing techniques, alongside the examination of relevant hardware and software tools. Following this, a comprehensive understanding of \gls{gnss} systems, with a particular emphasis on the \gls{gps} system, was acquired. Initially, authentic \gls{gps} signals were captured using both commercial hardware and a software-defined radio (\gls{sdr}). This was followed by the generation of simulated \gls{gps} signals, enabling the spoofing of the same commercial receiver for further investigation and analysis.

The report is structured as follows, the first chapter gives a brief overview of existing systems as a state of the art. The second chapter provides a detailed explanation of the technical concepts that are essential to understanding the rest of the document. The third, fourth and fifth chapters delve into the specifics of the \gls{gps} system, exploring its characteristics and limitations. Finally, the sixth chapter presents the different experimentations that were conducted to achieve the project's objectives.